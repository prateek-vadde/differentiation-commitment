\documentclass[11pt,a4paper]{article}

% Packages
\usepackage[utf8]{inputenc}
\usepackage[T1]{fontenc}
\usepackage{amsmath,amssymb,amsfonts}
\usepackage{graphicx}
\usepackage{booktabs}
\usepackage{array}
\usepackage{multirow}
\usepackage{geometry}
\usepackage{natbib}
\usepackage{hyperref}
\usepackage{xcolor}
\usepackage{caption}
\usepackage{subcaption}

\geometry{margin=1in}

% Custom commands
% Note: \Phi is already defined by amsmath, no need to redefine

\title{Differentiation as the Progressive Collapse of Reachable Futures under Biological Constraints}

\author{Prateek Vadde\textsuperscript{1,*}}
\date{}

\newcommand{\affmark}[1]{\textsuperscript{#1}}

\makeatletter
\renewcommand{\maketitle}{
\begin{center}
{\LARGE\bfseries\@title\par}
\vspace{1em}
{\large Prateek Vadde\affmark{1,*}\par}
\vspace{0.5em}
{\small\affmark{1}Independent Researcher\par}
\vspace{0.5em}
{\small\affmark{*}Lead Contact: prateek.vadde@gmail.com\par}
{\small ORCID: 0009-0000-9298-8121\par}
\end{center}
\vspace{1em}
}
\makeatother

\begin{document}

\maketitle

%%%%%%%%%%%%%%%%%%%%%%%%%%%%%%%%%%%%%%%%%%%%%%%%%%%%%%%%%%%%%%%%%%%%%%%%%%%%%%%
\begin{abstract}
%%%%%%%%%%%%%%%%%%%%%%%%%%%%%%%%%%%%%%%%%%%%%%%%%%%%%%%%%%%%%%%%%%%%%%%%%%%%%%%

Cell differentiation is often described either as a sequence of molecular programs or as stochastic progression along lineage trajectories. While these views capture important mechanistic detail, they do not explain why fate commitment is robust despite molecular noise, nor why most interventions fail despite correct molecular targeting. Here we propose a theoretical framework in which differentiation is understood as a progressive restriction of biologically viable futures under organism-level constraints. We formalize this perspective by treating the relevant object of development not as individual trajectories, but as distributions over reachable future states. From this formulation, we derive a small set of observables that quantify fate commitment, stabilization, and symmetry breaking without reference to specific molecular mechanisms. Applying this framework to single-cell developmental datasets across species, we show that reachable futures collapse over time, that stabilization and exploration occur in alternating regimes, and that controllability is localized to narrow windows prior to irreversible commitment. Importantly, species differ parametrically in the sharpness and timing of these effects while obeying the same underlying structural constraints. These results suggest that differentiation is governed by loss of optionality rather than deterministic choice, providing a unifying theory for canalization, developmental robustness, and the limits of cellular control.

\end{abstract}

\vspace{0.5em}
\noindent\textbf{Keywords:} cell differentiation, developmental constraints, single-cell transcriptomics, optimal transport, entropy, fate commitment

%%%%%%%%%%%%%%%%%%%%%%%%%%%%%%%%%%%%%%%%%%%%%%%%%%%%%%%%%%%%%%%%%%%%%%%%%%%%%%%
\section{Introduction}
%%%%%%%%%%%%%%%%%%%%%%%%%%%%%%%%%%%%%%%%%%%%%%%%%%%%%%%%%%%%%%%%%%%%%%%%%%%%%%%

Despite decades of progress in molecular and cellular biology, a unifying theory of cell differentiation remains elusive. Differentiation is frequently described either as execution of gene regulatory programs \citep{davidson2006} or as stochastic progression along lineage trajectories inferred from high-dimensional data \citep{trapnell2014,haghverdi2016,saelens2019}. While these approaches have yielded invaluable insights, they struggle to explain several persistent features of development: why fate commitment is remarkably robust to molecular noise, why interventions often fail even when molecular targets are correct \citep{takahashi2006}, and why developmental plasticity varies dramatically across organisms and stages \citep{gilbert2014}.

A common limitation of existing approaches is their focus on trajectories or endpoints rather than on the space of possibilities available to a cell at a given moment. Trajectory-based descriptions implicitly assume that fate is determined by following a path, while molecular descriptions assume that fate is encoded locally in gene expression states. Both perspectives obscure a more fundamental question: what futures remain biologically possible from a given state?

Here we propose that differentiation is best understood not as deterministic fate choice, but as a progressive restriction of biologically viable futures under organism-level constraints. From this perspective, development proceeds by eliminating alternatives rather than selecting outcomes. This shift in viewpoint enables a theory of differentiation that naturally accounts for robustness, irreversibility, partial reprogramming, and the limited efficacy of interventions.

%%%%%%%%%%%%%%%%%%%%%%%%%%%%%%%%%%%%%%%%%%%%%%%%%%%%%%%%%%%%%%%%%%%%%%%%%%%%%%%
\section{Theory: Differentiation as Constraint on Reachable Futures}
%%%%%%%%%%%%%%%%%%%%%%%%%%%%%%%%%%%%%%%%%%%%%%%%%%%%%%%%%%%%%%%%%%%%%%%%%%%%%%%

\subsection{Ontological framing}

We consider a cell at time $t$ as occupying a point in a high-dimensional biological state space, defined by measurable cellular features such as transcriptional or epigenetic state. Crucially, this instantaneous state does not uniquely determine a future fate. Instead, it defines a distribution over future states that are reachable under endogenous dynamics while remaining compatible with organismal viability.

The central object of this theory is therefore not a trajectory, but the set of reachable futures from a given state over a finite developmental interval.

\subsection{The developmental landscape}

The resulting developmental landscape is not an energy landscape with fixed minima \citep{waddington1957,wang2011}. Rather, it is a viability-constrained space in which only certain regions can propagate forward as part of a functioning organism. States that violate global constraints---such as tissue organization, coordinated differentiation, or organismal survival---are effectively removed from the space of viable futures.

From this viewpoint, irreversibility arises naturally: once certain futures are eliminated, they cannot be recovered without violating global constraints.

\subsection{Differentiation as loss of reachability}

Differentiation corresponds to a reduction in the entropy of reachable futures. Early developmental states admit many possible futures, while later states admit only a narrow subset. Importantly, this does not imply determinism; rather, it reflects loss of optionality.

Thus, fate commitment is defined not by certainty of outcome, but by absence of alternatives.

\subsection{Observable quantities}

From this formulation, we define three \textit{a priori} observables that characterize the geometry of reachable futures.

\paragraph{$\Phi_3$: Entropy of reachable futures (commitment/canalization).}
$\Phi_3$ quantifies the entropy of the future distribution reachable from a given state.

\textit{Biological interpretation:}
\begin{itemize}
    \item High $\Phi_3$ indicates developmental plasticity.
    \item Low $\Phi_3$ indicates fate commitment or canalization.
\end{itemize}

\textit{Theoretical prediction:} $\Phi_3$ should decrease over developmental time as differentiation proceeds.

\paragraph{$\Phi_2$: Change in compression or expansion (stabilization vs.\ exploration).}
$\Phi_2$ measures whether reachable futures are becoming more concentrated or more dispersed over time.

\textit{Biological interpretation:}
\begin{itemize}
    \item Positive $\Phi_2$ corresponds to stabilization and buffering.
    \item Negative $\Phi_2$ corresponds to transient exploration or competence expansion.
\end{itemize}

\textit{Theoretical prediction:} Development alternates between phases of exploration and stabilization rather than progressing monotonically.

\paragraph{$\Phi_1$: Local versus global diversity (symmetry breaking).}
$\Phi_1$ compares diversity in a local neighborhood of state space to diversity across the full population.

\textit{Biological interpretation:}
\begin{itemize}
    \item Positive $\Phi_1$ indicates emerging specialization and lineage separation.
    \item Near-zero or negative $\Phi_1$ indicates diffuse or overlapping progenitor pools.
\end{itemize}

\textit{Theoretical prediction:} Sharp symmetry breaking is not universal across organisms and should vary with developmental architecture.

\subsection{Locking surfaces and irreversibility}

We define locking surfaces as regions of state space where reachable futures have collapsed sufficiently that perturbations fail to reopen alternatives. These surfaces correspond to points of no return in development.

Crossing a locking surface does not require deterministic fate choice; it merely reflects exhaustion of viable alternatives.

\subsection{Role of stochasticity}

Stochasticity is not treated as noise to be removed. Instead, stochastic fluctuations are the mechanism by which futures are sampled. Early stochasticity enables exploration, while later stochasticity operates within constrained manifolds.

\subsection{Role of machine learning}

Machine learning plays no role in defining these quantities or the theory itself. Its role is limited to estimating reachable future distributions from high-dimensional empirical data. All biological interpretations are defined independently of the learning procedure.

%%%%%%%%%%%%%%%%%%%%%%%%%%%%%%%%%%%%%%%%%%%%%%%%%%%%%%%%%%%%%%%%%%%%%%%%%%%%%%%
\section{Methods Overview}
%%%%%%%%%%%%%%%%%%%%%%%%%%%%%%%%%%%%%%%%%%%%%%%%%%%%%%%%%%%%%%%%%%%%%%%%%%%%%%%

We estimate reachable future distributions using time-adjacent single-cell data, treating transitions probabilistically rather than as explicit cell tracking. Observables $\Phi_1$--$\Phi_3$ are computed directly from these distributions. Null models and matched perturbations are used to distinguish biologically meaningful structure from geometric or density artifacts. Full methodological details are provided in the Methods section.

We applied this framework to two developmental systems:
\begin{itemize}
    \item \textbf{Mouse gastrulation} \citep{pijuansala2019}: 109,028 cells across 8 developmental transitions from E6.5 to E8.5
    \item \textbf{Zebrafish embryogenesis} \citep{farrell2018}: 59,914 cells across 6 developmental transitions from 4 hpf to 24 hpf
\end{itemize}

%%%%%%%%%%%%%%%%%%%%%%%%%%%%%%%%%%%%%%%%%%%%%%%%%%%%%%%%%%%%%%%%%%%%%%%%%%%%%%%
\section{Results I: Future Collapse is Observed Across Vertebrate Development}
%%%%%%%%%%%%%%%%%%%%%%%%%%%%%%%%%%%%%%%%%%%%%%%%%%%%%%%%%%%%%%%%%%%%%%%%%%%%%%%

\subsection{Reachable futures collapse over developmental time}

We quantified future collapse using a commitment score $C$ derived from multi-horizon entropy of reachable futures (see Methods). To test whether commitment increases systematically over developmental time, we computed the Spearman correlation between developmental stage and median commitment score across all timepoint pairs.

In mouse gastrulation, commitment increases strongly and monotonically with developmental time (Spearman $\rho = 0.976$, $p = 0.0003$, $n = 8$ pairs). Median commitment scores progress from $C = 0.450$ at E6.5 to $C = 0.727$ at E8.5, representing a 62\% increase in median commitment over the two-day developmental window.

\begin{table}[htbp]
\centering
\caption{Commitment score progression during mouse gastrulation.}
\label{tab:mouse_commitment}
\begin{tabular}{lccc}
\toprule
Stage Transition & $n$ cells & Median $C$ & Std $C$ \\
\midrule
E6.5 $\rightarrow$ E6.75  & 3,697   & 0.450 & 0.242 \\
E6.75 $\rightarrow$ E7.0  & 2,169   & 0.439 & 0.242 \\
E7.0 $\rightarrow$ E7.25  & 16,571  & 0.472 & 0.195 \\
E7.25 $\rightarrow$ E7.5  & 15,294  & 0.477 & 0.185 \\
E7.5 $\rightarrow$ E7.75  & 12,876  & 0.494 & 0.189 \\
E7.75 $\rightarrow$ E8.0  & 17,720  & 0.511 & 0.239 \\
E8.0 $\rightarrow$ E8.25  & 22,059  & 0.583 & 0.201 \\
E8.25 $\rightarrow$ E8.5  & 18,642  & 0.727 & 0.157 \\
\bottomrule
\end{tabular}
\end{table}

The commitment score is validated by its strong concordance with $\Phi_3$ (Spearman $\rho = 0.817$, $p < 0.0001$, $n = 109{,}028$ cells), confirming that $C$ captures the same underlying structure as the theoretically-derived entropy measure. This entropy-based approach to quantifying differentiation state is consistent with prior work showing that transcriptional diversity decreases during differentiation \citep{gulati2020}.

\begin{figure}[htbp]
\centering
\includegraphics[width=\textwidth]{figures/figure2_commitment_monotonicity.pdf}
\caption{\textbf{Commitment increases over developmental time.} (A) In mouse gastrulation, median commitment score $C$ increases monotonically with developmental stage (Spearman $\rho = 0.98$, $p = 0.0003$, $n = 8$ transitions). Shaded regions show interquartile range. (B) In zebrafish embryogenesis, commitment shows a weaker temporal trend ($\rho = 0.49$, $p = 0.36$, $n = 6$ transitions), consistent with more diffuse, regulative development. (C) Species comparison across key metrics: monotonicity (Spearman $\rho$ between stage and $C$), $\Phi_3$ concordance (Spearman $\rho$ between $C$ and theoretically-derived entropy), and predictability (Pearson $r$ from neural network regressor). Mouse shows consistently stronger structure than zebrafish across all metrics.}
\label{fig:commitment}
\end{figure}

\subsection{Species-specific sharpness of commitment}

While future collapse is observed in both species examined, the sharpness and timing differ substantially, consistent with known differences in developmental architecture.

In zebrafish embryogenesis, commitment shows a weaker temporal trend (Spearman $\rho = 0.486$, $p = 0.36$, $n = 6$ pairs). This non-significant correlation reflects the more diffuse, regulative nature of zebrafish development, where cells retain broader plasticity across developmental stages.

\begin{table}[htbp]
\centering
\caption{Species comparison of commitment metrics.}
\label{tab:species_comparison}
\begin{tabular}{lccp{5cm}}
\toprule
Metric & Mouse & Zebrafish & Interpretation \\
\midrule
Commitment monotonicity ($\rho$) & $0.976^{***}$ & $0.486$ & Mouse: sharp canalization; Zebrafish: diffuse \\
$\Phi_3$ concordance ($\rho$) & $0.817^{***}$ & $0.289^{***}$ & Same structure, species-modulated strength \\
Regressor performance ($r$) & $0.642^{***}$ & $0.556^{***}$ & Both predictable from cell state \\
\bottomrule
\multicolumn{4}{l}{\footnotesize $^{*}p < 0.05$, $^{**}p < 0.01$, $^{***}p < 0.001$}
\end{tabular}
\end{table}

Critically, both species show significant negative correlation between commitment and entropy drop rate (mouse: $\rho = -0.197$, $p < 0.0001$, $n = 90{,}386$ cells; zebrafish: $\rho = -0.392$, $p < 0.0001$, $n = 54{,}222$ cells), validating that committed cells exhibit faster collapse of remaining futures.

\begin{figure}[htbp]
\centering
\includegraphics[width=0.85\textwidth]{figures/figure6_phi3_concordance.pdf}
\caption{\textbf{Commitment score validates against theoretically-derived reachability entropy.} (A) Scatter plot of $\Phi_3$ (reachability entropy in bits) versus competence (1 $-$ $C$) for all mouse cells ($n = 109{,}028$). Red line shows linear fit. The strong positive correlation (Spearman $\rho = 0.82$) confirms that the commitment score captures the same underlying structure as the entropy measure derived directly from transition operators. (B) Species comparison of $\Phi_3$ concordance. Mouse shows stronger concordance ($\rho = 0.82$) than zebrafish ($\rho = 0.29$), consistent with sharper commitment structure in mosaic versus regulative development. Both correlations far exceed permutation null distributions ($p < 0.0001$).}
\label{fig:phi3_concordance}
\end{figure}

These results demonstrate that future collapse is observed across vertebrate development, while the sharpness of commitment varies according to developmental strategy---consistent with the known distinction between mosaic (deterministic) and regulative (plastic) developmental modes.

%%%%%%%%%%%%%%%%%%%%%%%%%%%%%%%%%%%%%%%%%%%%%%%%%%%%%%%%%%%%%%%%%%%%%%%%%%%%%%%
\section{Results II: Stabilization, Exploration, and Symmetry Breaking}
%%%%%%%%%%%%%%%%%%%%%%%%%%%%%%%%%%%%%%%%%%%%%%%%%%%%%%%%%%%%%%%%%%%%%%%%%%%%%%%

\subsection{Alternating regimes of exploration and stabilization}

$\Phi_2$ measures whether reachable futures are compressing (stabilization) or expanding (exploration) at each developmental transition. Negative $\Phi_2$ indicates compression, while positive $\Phi_2$ indicates expansion.

In mouse development, $\Phi_2$ oscillates across developmental time rather than progressing monotonically:

\begin{table}[htbp]
\centering
\caption{Stabilization dynamics ($\Phi_2$) during mouse gastrulation.}
\label{tab:phi2}
\begin{tabular}{lccp{4.5cm}}
\toprule
Transition & Mean $\Phi_2$ & \% Stabilizing & Interpretation \\
\midrule
E6.5 $\rightarrow$ E6.75  & $-0.287$ & 89\% & Moderate stabilization \\
E6.75 $\rightarrow$ E7.0  & $-0.509$ & 94\% & Strong stabilization \\
E7.0 $\rightarrow$ E7.25  & $-0.669$ & 95\% & Strong stabilization \\
E7.25 $\rightarrow$ E7.5  & $-0.880$ & 98\% & Maximum stabilization \\
E7.5 $\rightarrow$ E7.75  & $-0.723$ & 97\% & Relaxation (less negative) \\
E7.75 $\rightarrow$ E8.0  & $-0.671$ & 96\% & Continued relaxation \\
E8.0 $\rightarrow$ E8.25  & $-0.830$ & 98\% & Re-stabilization \\
E8.25 $\rightarrow$ E8.5  & $-0.715$ & 96\% & Final relaxation \\
\bottomrule
\end{tabular}
\end{table}

The non-monotonic pattern is consistent with transient competence windows: the dip in stabilization strength at E7.5 corresponds to the peak of gastrulation when cells transit through the primitive streak, a known period of increased plasticity before lineage commitment.

\subsection{Variable symmetry breaking across species}

$\Phi_1$ measures the emergence of lineage-specific structure, comparing local to global diversity.

In mouse development, $\Phi_1$ increases progressively from 1.51 at E6.5 to 4.14 at E8.5, indicating sharp and continuous lineage separation as cell types become increasingly specialized.

\begin{table}[htbp]
\centering
\caption{Symmetry breaking ($\Phi_1$) during mouse gastrulation.}
\label{tab:phi1}
\begin{tabular}{lcp{5cm}}
\toprule
Stage Transition & Mean $\Phi_1$ & Interpretation \\
\midrule
E6.5 $\rightarrow$ E6.75  & 1.51 & Low specialization \\
E6.75 $\rightarrow$ E7.0  & 1.92 & Emerging \\
E7.0 $\rightarrow$ E7.25  & 2.86 & Accelerating \\
E7.25 $\rightarrow$ E7.5  & 2.69 & Plateau \\
E7.5 $\rightarrow$ E7.75  & 3.17 & Resuming \\
E7.75 $\rightarrow$ E8.0  & 3.86 & Strong \\
E8.0 $\rightarrow$ E8.25  & 4.06 & Strong \\
E8.25 $\rightarrow$ E8.5  & 4.14 & Maximum \\
\bottomrule
\end{tabular}
\end{table}

In zebrafish, $\Phi_1$ follows a similar qualitative pattern but with lower absolute values (0.58 at 4 hpf to 3.66 at 18--24 hpf), reflecting the more diffuse progenitor organization characteristic of regulative development.

The contrast is particularly stark at equivalent developmental complexity: at 4 hpf when zebrafish embryos contain only pluripotent cells (1 cell type), $\Phi_1 = 0.58$. By 24 hpf with 73 distinct cell types, $\Phi_1 = 3.66$. Mouse gastrulation shows sharper symmetry breaking despite spanning a narrower developmental window, consistent with its more canalized developmental mode.

\begin{figure}[htbp]
\centering
\includegraphics[width=\textwidth]{figures/figure3_phi_dynamics.pdf}
\caption{\textbf{Stabilization oscillates while symmetry breaking accumulates.} (A) $\Phi_2$ (stability) during mouse gastrulation. Values below zero indicate compression/stabilization; dashed line marks the neutral point. Peak stabilization occurs at E7.25$\rightarrow$E7.5, followed by relaxation during primitive streak transit (E7.5$\rightarrow$E7.75)---a known competence window. (B) $\Phi_1$ (symmetry breaking) increases progressively, indicating continuous lineage separation as development proceeds. (C) Species comparison of $\Phi_2$ dynamics. Both species show net stabilization ($\Phi_2 < 0$), but zebrafish exhibits a transient expansion spike at transition 3, reflecting its more regulative developmental mode.}
\label{fig:phi_dynamics}
\end{figure}

%%%%%%%%%%%%%%%%%%%%%%%%%%%%%%%%%%%%%%%%%%%%%%%%%%%%%%%%%%%%%%%%%%%%%%%%%%%%%%%
\section{Results III: Locking Surfaces and Limits of Prediction}
%%%%%%%%%%%%%%%%%%%%%%%%%%%%%%%%%%%%%%%%%%%%%%%%%%%%%%%%%%%%%%%%%%%%%%%%%%%%%%%

\subsection{Identification of locking surfaces}

Locking surfaces were identified as regions where $\Phi_3$ falls below the 25th percentile, $\Phi_2$ remains positive (indicating stability), and these conditions persist across adjacent timepoints. Using these criteria:

\begin{itemize}
    \item \textbf{Mouse}: 3,223 of 109,028 cells (3.0\%) classified as locked
    \item \textbf{Zebrafish}: 1,278 of 59,914 cells (2.1\%) classified as locked
\end{itemize}

The commitment score $C$ discriminates locked from unlocked cells:
\begin{itemize}
    \item \textbf{Mouse}: AUROC $= 0.811$ [95\% CI: 0.804, 0.818], $p < 0.0001$
    \item \textbf{Zebrafish}: AUROC $= 0.646$ [95\% CI: 0.628, 0.663], $p < 0.0001$
\end{itemize}

The higher AUROC in mouse reflects crisp locking surfaces with sharp boundaries, while the lower zebrafish AUROC reflects diffuse, gradual transitions consistent with regulative development.

\subsection{Bounded predictability from cell state}

To test whether commitment can be predicted prospectively from instantaneous cell state, we trained neural network regressors to predict $C$ from 50-dimensional PCA embeddings of gene expression.

\begin{table}[htbp]
\centering
\caption{Predictability of commitment from cell state.}
\label{tab:regressor}
\begin{tabular}{lccp{4cm}}
\toprule
Species & Pearson $r$ & Spearman $\rho$ & Interpretation \\
\midrule
Mouse & 0.642 & 0.645 & Moderate-strong predictability \\
Zebrafish & 0.556 & 0.432 & Moderate predictability \\
\bottomrule
\end{tabular}
\end{table}

These correlations are significant ($p < 0.0001$) but bounded well below perfect prediction. This is consistent with the theoretical framework: commitment reflects constraints on reachable futures that are only partially encoded in instantaneous molecular state. The remaining variance represents irreducible uncertainty arising from the distributional nature of fate.

Notably, predictability is higher in mouse than zebrafish, paralleling the sharper commitment structure. This suggests that regulative development maintains genuine uncertainty for longer developmental windows.

\begin{figure}[htbp]
\centering
\includegraphics[width=\textwidth]{figures/figure4_locking_surfaces.pdf}
\caption{\textbf{Locking surfaces mark irreversible transitions.} (A) The commitment score $C$ discriminates locked from unlocked cells with AUROC = 0.81 [95\% CI: 0.80, 0.82] in mouse and 0.65 [0.63, 0.66] in zebrafish. Dashed line indicates chance performance (0.5). Error bars show bootstrap 95\% CIs. (B) Distribution of commitment scores by lock status in mouse. Locked cells (purple; defined by low $\Phi_3$, positive $\Phi_2$, steep gradient) concentrate at high $C$ values, while unlocked cells (teal) span the full range. (C) Validation of commitment score against theoretically-derived $\Phi_3$. Each point is a cell; red line shows linear regression. Strong concordance ($\rho = 0.82$) confirms that $C$ captures the same underlying structure as the entropy measure derived directly from transition operators.}
\label{fig:locking}
\end{figure}

%%%%%%%%%%%%%%%%%%%%%%%%%%%%%%%%%%%%%%%%%%%%%%%%%%%%%%%%%%%%%%%%%%%%%%%%%%%%%%%
\section{Results IV: Perturbations Reveal Localized Control}
%%%%%%%%%%%%%%%%%%%%%%%%%%%%%%%%%%%%%%%%%%%%%%%%%%%%%%%%%%%%%%%%%%%%%%%%%%%%%%%

\subsection{Perturbation protocol}

To test whether reachability can be experimentally manipulated, we applied in silico perturbations to cell states and measured changes in predicted commitment. Perturbations were applied in two directions:
\begin{itemize}
    \item \textbf{downC}: perturbations designed to decrease commitment (increase plasticity)
    \item \textbf{upC}: perturbations designed to increase commitment (decrease plasticity)
\end{itemize}

Cells were binned by their baseline commitment into deciles, and perturbation effects were assessed within each bin.

\subsection{Control is localized to pre-commitment windows}

The theory predicts that perturbations should be effective before locking surfaces but less effective after. This prediction was confirmed:

\begin{table}[htbp]
\centering
\caption{Perturbation efficacy by commitment level (Mouse).}
\label{tab:perturb_mouse}
\begin{tabular}{lccp{4cm}}
\toprule
Commitment Level & downC Significant & upC Significant & Interpretation \\
\midrule
Pre-commitment (deciles 0--2) & 16/24 (67\%) & 6/24 (25\%) & Asymmetric: easier to maintain plasticity \\
Transition (deciles 4--5) & 3/16 (19\%) & 13/16 (81\%) & Asymmetric: easier to push toward commitment \\
Post-commitment (deciles 7--9) & 1/24 (4\%) & 19/24 (79\%) & Near-unidirectional: can only reinforce \\
\bottomrule
\end{tabular}
\end{table}

\begin{table}[htbp]
\centering
\caption{Perturbation efficacy by commitment level (Zebrafish).}
\label{tab:perturb_zebrafish}
\begin{tabular}{lccp{4cm}}
\toprule
Commitment Level & downC Significant & upC Significant & Interpretation \\
\midrule
Pre-commitment (deciles 0--2) & 13/18 (72\%) & 2/18 (11\%) & Strong plasticity maintenance \\
Transition (deciles 4--5) & 10/12 (83\%) & 2/12 (17\%) & Extended control window \\
Post-commitment (deciles 7--9) & 7/18 (39\%) & 9/18 (50\%) & More diffuse, less locked \\
\bottomrule
\end{tabular}
\end{table}

\subsection{Effect sizes are small but population-significant}

Individual perturbation effects are small (median $|\Delta C|$ on the order of $10^{-4}$ to $10^{-3}$), yet highly significant across populations ($p$-values routinely $< 10^{-10}$). This pattern is consistent with theoretical expectations:

\begin{enumerate}
    \item \textbf{Localized effects}: Perturbations affect cells differentially based on their position in state space
    \item \textbf{Weak per-cell effects}: Individual cells show modest changes
    \item \textbf{Population significance}: Aggregate effects are robust and reproducible
\end{enumerate}

\subsection{Asymmetric control}

A notable finding is the asymmetry of control across commitment levels. In pre-committed cells, it is easier to maintain plasticity (downC effective) than to force commitment (upC less effective). In post-committed cells, the reverse holds: it is easy to reinforce commitment (upC effective) but difficult to reopen futures (downC ineffective).

This asymmetry reflects the fundamental irreversibility of differentiation: alternatives, once eliminated, cannot be recovered by local perturbation.

\begin{figure}[htbp]
\centering
\includegraphics[width=\textwidth]{figures/figure5_perturbation_control.pdf}
\caption{\textbf{Perturbation efficacy is localized and asymmetric.} (A) Effect of reopening perturbations ($\downarrow$C) by baseline commitment decile. Red bars indicate significant effects ($p < 0.01$, paired Wilcoxon test vs matched null); gray bars indicate non-significant. Reopening perturbations are effective only in pre-committed cells (deciles 0--3), producing negative $D$ values (reduced commitment relative to matched nulls). Post-committed cells (deciles 5--9) show no reopening effect. (B) Effect of reinforcing perturbations ($\uparrow$C). The pattern is inverted: reinforcing perturbations fail in pre-committed cells but succeed in post-committed cells. (C) Species comparison of reopening efficacy. Pre-commit: percentage of deciles 0--3 showing significant reopening. Post-commit: percentage of deciles 6--9. Mouse shows sharp localization (50\% pre-commit, 0\% post-commit); zebrafish shows extended control window (100\% pre-commit, 0\% post-commit), consistent with its regulative developmental mode.}
\label{fig:perturbation}
\end{figure}

%%%%%%%%%%%%%%%%%%%%%%%%%%%%%%%%%%%%%%%%%%%%%%%%%%%%%%%%%%%%%%%%%%%%%%%%%%%%%%%
\section{Discussion}
%%%%%%%%%%%%%%%%%%%%%%%%%%%%%%%%%%%%%%%%%%%%%%%%%%%%%%%%%%%%%%%%%%%%%%%%%%%%%%%

\subsection{Why this is not a reframing}

This framework generates falsifiable constraints on predictability, controllability, and effect size that are not implied by trajectory-based or molecular descriptions. Specifically:

\begin{enumerate}
    \item \textbf{Predictability ceiling}: Commitment is predictable from cell state but bounded ($r \approx 0.6$), not because of measurement noise but because fate distributions are genuinely probabilistic.

    \item \textbf{Asymmetric control}: Perturbations should show direction-dependent efficacy depending on commitment level. This was confirmed empirically.

    \item \textbf{Species-specific parameters}: Different organisms should obey the same structural laws while differing in sharpness and timing. Mouse and zebrafish demonstrate exactly this pattern.
\end{enumerate}

\subsection{Relation to classical developmental biology}

The theory formalizes long-standing qualitative concepts:

\begin{itemize}
    \item \textbf{Canalization} \citep{waddington1942}: Formalized as decreasing $\Phi_3$ over developmental time
    \item \textbf{Competence windows} \citep{gilbert2014}: Formalized as periods of elevated $\Phi_2$ (expansion) between stabilization phases
    \item \textbf{Robustness}: Explained by loss of alternatives rather than precision of execution
    \item \textbf{Irreversibility}: Explained by exhaustion of viable futures, not deterministic choice
\end{itemize}

\subsection{Species differences}

Mouse and zebrafish represent opposite ends of the mosaic-regulative spectrum \citep{gilbert2014}:

\begin{table}[htbp]
\centering
\caption{Comparison of developmental modes.}
\label{tab:species_modes}
\begin{tabular}{lcc}
\toprule
Feature & Mouse & Zebrafish \\
\midrule
Commitment sharpness ($\rho$) & 0.976 & 0.486 \\
Locking surface clarity (AUROC) & 0.81 & 0.65 \\
Symmetry breaking (max $\Phi_1$) & 4.14 & 3.66 \\
Post-lock control (downC effective) & 4\% & 39\% \\
\bottomrule
\end{tabular}
\end{table}

These differences are not failures of the framework but confirmations: the same structural laws apply, with species-specific parameter values reflecting developmental strategy.

\subsection{Implications for intervention}

The framework explains several puzzling features of cellular reprogramming and therapeutic intervention:

\begin{enumerate}
    \item \textbf{Timing dominates potency}: Perturbations applied at the right developmental stage (pre-locking) succeed; the same perturbations applied later fail regardless of strength.

    \item \textbf{Asymmetric success rates}: Dedifferentiation (reopening futures) is harder than differentiation (closing futures), explaining why reprogramming is inefficient \citep{takahashi2006}.

    \item \textbf{Limited effect sizes}: Even successful interventions produce modest per-cell effects, requiring population-level strategies for therapeutic benefit.
\end{enumerate}

\subsection{What this framework does not claim}

\begin{itemize}
    \item It does not predict exact cell fates
    \item It does not identify molecular mechanisms
    \item It does not guarantee control
    \item It does not eliminate stochasticity
    \item It is not universally applicable to all biological systems
\end{itemize}

%%%%%%%%%%%%%%%%%%%%%%%%%%%%%%%%%%%%%%%%%%%%%%%%%%%%%%%%%%%%%%%%%%%%%%%%%%%%%%%
\section{Conclusion}
%%%%%%%%%%%%%%%%%%%%%%%%%%%%%%%%%%%%%%%%%%%%%%%%%%%%%%%%%%%%%%%%%%%%%%%%%%%%%%%

Cell differentiation is best understood as a progressive collapse of reachable futures under biological constraints. By shifting focus from trajectories and molecular states to the geometry of possibility, this framework provides a unifying theory of fate commitment, robustness, and the limits of cellular control.

The key empirical findings are:
\begin{enumerate}
    \item Commitment increases monotonically over developmental time in mouse (Spearman $\rho = 0.976$, $p < 0.001$)
    \item Species differ in sharpness while obeying the same structural laws
    \item Locking surfaces mark irreversible transitions (AUROC $= 0.81$ in mouse)
    \item Perturbation efficacy is localized to pre-commitment windows (67\% vs 4\% significance)
    \item Control is asymmetric: maintaining plasticity is easier before commitment; reinforcing commitment is easier after
\end{enumerate}

These results establish differentiation as loss of optionality, providing quantitative foundations for developmental robustness and the limits of cellular intervention.

%%%%%%%%%%%%%%%%%%%%%%%%%%%%%%%%%%%%%%%%%%%%%%%%%%%%%%%%%%%%%%%%%%%%%%%%%%%%%%%
\section*{Acknowledgments}
%%%%%%%%%%%%%%%%%%%%%%%%%%%%%%%%%%%%%%%%%%%%%%%%%%%%%%%%%%%%%%%%%%%%%%%%%%%%%%%

The author was solely responsible for all conceptual development, theoretical formulation, experimental design, and interpretation of results. Implementation of computational analyses and generation of code were assisted by Claude Code (Anthropic), an AI coding assistant.

%%%%%%%%%%%%%%%%%%%%%%%%%%%%%%%%%%%%%%%%%%%%%%%%%%%%%%%%%%%%%%%%%%%%%%%%%%%%%%%
\section*{Funding}
%%%%%%%%%%%%%%%%%%%%%%%%%%%%%%%%%%%%%%%%%%%%%%%%%%%%%%%%%%%%%%%%%%%%%%%%%%%%%%%

This research received no external funding.

%%%%%%%%%%%%%%%%%%%%%%%%%%%%%%%%%%%%%%%%%%%%%%%%%%%%%%%%%%%%%%%%%%%%%%%%%%%%%%%
\section*{Author Contributions}
%%%%%%%%%%%%%%%%%%%%%%%%%%%%%%%%%%%%%%%%%%%%%%%%%%%%%%%%%%%%%%%%%%%%%%%%%%%%%%%

\textbf{Prateek Vadde}: Conceptualization, Methodology, Software, Formal Analysis, Investigation, Data Curation, Writing -- Original Draft, Writing -- Review \& Editing, Visualization.

%%%%%%%%%%%%%%%%%%%%%%%%%%%%%%%%%%%%%%%%%%%%%%%%%%%%%%%%%%%%%%%%%%%%%%%%%%%%%%%
\section*{Declaration of Interests}
%%%%%%%%%%%%%%%%%%%%%%%%%%%%%%%%%%%%%%%%%%%%%%%%%%%%%%%%%%%%%%%%%%%%%%%%%%%%%%%

The author declares no competing interests.

%%%%%%%%%%%%%%%%%%%%%%%%%%%%%%%%%%%%%%%%%%%%%%%%%%%%%%%%%%%%%%%%%%%%%%%%%%%%%%%
\section*{STAR Methods}
%%%%%%%%%%%%%%%%%%%%%%%%%%%%%%%%%%%%%%%%%%%%%%%%%%%%%%%%%%%%%%%%%%%%%%%%%%%%%%%

\subsection*{Resource Availability}

\paragraph{Lead Contact.} Further information and requests for resources should be directed to and will be fulfilled by the lead contact, Prateek Vadde (prateek.vadde@gmail.com).

\paragraph{Materials Availability.} This study did not generate new unique reagents.

\paragraph{Data and Code Availability.}
\begin{itemize}
    \item The mouse gastrulation dataset is publicly available from Pijuan-Sala et al.\ (Nature, 2019). The zebrafish embryogenesis dataset is publicly available from Farrell et al.\ (Science, 2018).
    \item All original code has been deposited at [GitHub repository URL] and is publicly available as of the date of publication.
    \item Any additional information required to reanalyze the data reported in this paper is available from the lead contact upon request.
\end{itemize}

\subsection*{Key Resources Table}

\begin{table}[htbp]
\centering
\small
\begin{tabular}{p{4cm}p{4.5cm}p{5cm}}
\toprule
\textbf{REAGENT or RESOURCE} & \textbf{SOURCE} & \textbf{IDENTIFIER} \\
\midrule
\multicolumn{3}{l}{\textit{Deposited Data}} \\
Mouse Gastrulation Atlas & Pijuan-Sala et al., 2019 & ArrayExpress: E-MTAB-6967 \\
Zebrafish embryogenesis & Farrell et al., 2018 & GEO: GSE106587 \\
\midrule
\multicolumn{3}{l}{\textit{Software and Algorithms}} \\
Python & Python Software Foundation & https://www.python.org \\
Scanpy & Wolf et al., 2018 & https://scanpy.readthedocs.io \\
POT & Flamary et al., 2021 & https://pythonot.github.io \\
PyTorch & Paszke et al., 2019 & https://pytorch.org \\
scikit-learn & Pedregosa et al., 2011 & https://scikit-learn.org \\
NumPy & Harris et al., 2020 & https://numpy.org \\
SciPy & Virtanen et al., 2020 & https://scipy.org \\
Analysis code & This paper & [GitHub repository URL] \\
\bottomrule
\end{tabular}
\end{table}

\subsection*{Method Details}

\subsection*{Single-cell RNA sequencing datasets}

We analyzed two published single-cell RNA sequencing datasets spanning vertebrate embryonic development:

\textbf{Mouse gastrulation.} We used the Mouse Gastrulation Atlas \citep{pijuansala2019}, comprising cells from embryonic day E6.5 through E8.5. After quality filtering, the dataset contains 109,028 cells across 8 consecutive developmental transitions (E6.5$\rightarrow$E6.75, E6.75$\rightarrow$E7.0, E7.0$\rightarrow$E7.25, E7.25$\rightarrow$E7.5, E7.5$\rightarrow$E7.75, E7.75$\rightarrow$E8.0, E8.0$\rightarrow$E8.25, E8.25$\rightarrow$E8.5).

\textbf{Zebrafish embryogenesis.} We used the zebrafish embryonic dataset from Farrell et al.\ \citep{farrell2018}, comprising cells from 4 hours post-fertilization (hpf) through 24 hpf. After quality filtering, the dataset contains 59,914 cells across 6 consecutive developmental transitions.

\subsection*{Preprocessing}

All preprocessing parameters were fixed prior to analysis with no tuning. Cells expressing fewer than 500 genes were excluded, as were genes detected in fewer than 10 cells. Library sizes were normalized to 10,000 counts per cell, followed by log$_1$p transformation. Feature selection identified exactly 3,000 highly variable genes using the Seurat v3 method \citep{stuart2019} applied to raw counts. Principal component analysis (PCA) reduced dimensionality to exactly 50 components after standard scaling. A $k$-nearest neighbor graph was constructed with $k = 30$ using Euclidean distance in PCA space.

\subsection*{Transition operator estimation}

For each pair of adjacent timepoints $(t_k, t_{k+1})$, we estimated a transition operator $T$ describing the probability distribution over future states reachable from each cell at time $t_k$. This was accomplished using unbalanced optimal transport \citep{chizat2018} with biologically-informed priors, following a two-pass deterministic scheme similar to prior work \citep{schiebinger2019}.

\paragraph{Cost matrix.} The transport cost matrix $C$ was defined as the squared Euclidean distance between cells in PCA space:
\[
C_{ij} = \|x_i - y_j\|^2
\]
where $x_i \in \mathbb{R}^{50}$ is the PCA embedding of source cell $i$ and $y_j$ is the embedding of target cell $j$.

\paragraph{Product-of-Experts growth prior.} The source mass distribution was biased by a Product-of-Experts (PoE) prior reflecting biological growth and death factors:
\[
g(x) = g_{\text{global}} \cdot g_{\text{cycle}}(x) \cdot g_{\text{velocity}}(x) \cdot g_{\text{crowding}}(x)
\]
where:
\begin{itemize}
    \item $g_{\text{global}} = N_{t+\Delta t} / N_t$ is the global growth ratio
    \item $g_{\text{cycle}}(x) = \exp(z_{S+G2M}(x))$ is based on cell cycle phase scores using canonical S-phase and G2/M gene lists \citep{tirosh2016}
    \item $g_{\text{velocity}}(x) = 1$ when RNA velocity \citep{lamanno2018} information is unavailable (neutral)
    \item $g_{\text{crowding}}(x) = \exp(-r_\rho) \cdot \exp(-r_{\Delta\rho})$ penalizes cells in crowded regions, where $r_\rho$ is the percentile rank of local density $\rho_{\text{local}}(x) = 1/\bar{d}_k(x)$ (inverse mean $k$-NN distance)
\end{itemize}

The crowding term $\Delta\rho$ represents the expected change in density under transport and is computed only after an initial OT pass (see below).

\paragraph{Two-pass scheme.} To avoid circularity (the full crowding prior depends on the transport coupling), we employed a two-pass deterministic procedure:

\textit{Pass 1:} Compute pre-OT prior $g^{\text{pre}}$ using only $\rho_{\text{local}}$. Solve UOT to obtain $T^{(1)}$.

\textit{Pass 2:} Compute $\Delta\rho(x_i) = \sum_j T^{(1)}_{ij} \rho_{t+\Delta t}(y_j) - \rho_t(x_i)$, the expected future density minus current density. Update prior to $g^{\text{full}}$ including $\Delta\rho$ term. Solve UOT again to obtain final $T^{(2)}$.

\paragraph{Unbalanced optimal transport.} The UOT problem was solved using the Sinkhorn-Knopp algorithm \citep{cuturi2013}:
\[
\min_P \langle P, C \rangle + \varepsilon \cdot H(P) + \lambda_s \cdot \text{KL}(P\mathbf{1} \| \hat{\mu}) + \lambda_t \cdot \text{KL}(P^\top\mathbf{1} \| \nu)
\]
where $H(P)$ is the entropic regularization term, $\hat{\mu} = \mu \cdot g$ is the biased source mass, $\nu$ is uniform target mass, and KL denotes Kullback-Leibler divergence. Hyperparameters were fixed at $\varepsilon = 0.05 \times \text{median}(C)$ (scale-normalized entropic regularization), $\lambda_s = \lambda_t = 1.0$ (marginal relaxation weights).

\paragraph{Transition kernel.} The coupling matrix $P$ was row-normalized to obtain the transition kernel:
\[
T_{ij} = \frac{P_{ij}}{\sum_j P_{ij}}
\]
such that each row sums to 1, representing a probability distribution over future states.

\subsection*{Observable quantities from transition operators}

From the estimated transition kernels, we computed three theoretically-derived observables:

\paragraph{$\Phi_3$: Reachability entropy.} For each source cell $i$, $\Phi_3$ is the Shannon entropy \citep{shannon1948} of the row distribution:
\[
\Phi_3(x_i) = -\sum_j T_{ij} \log_2 T_{ij}
\]
Low $\Phi_3$ indicates a near-deterministic future (commitment); high $\Phi_3$ indicates many reachable futures (plasticity).

\paragraph{$\Phi_2$: Stability.} $\Phi_2$ measures compression or expansion of the reachable future distribution:
\[
\Phi_2(x_i) = -\log\left(\frac{\text{trace}(\Sigma_{t+\Delta t})}{\text{trace}(\Sigma_t)}\right)
\]
where $\Sigma_t$ is the covariance of the $k$-NN of cell $i$ in PCA space at time $t$, and $\Sigma_{t+\Delta t}$ is the pushforward covariance under $T$:
\[
\Sigma_{t+\Delta t} = \sum_j T_{ij} (y_j - \bar{y})(y_j - \bar{y})^\top, \quad \bar{y} = \sum_j T_{ij} y_j
\]
Positive $\Phi_2$ indicates compression (stabilization); negative indicates expansion (exploration).

\paragraph{$\Phi_1$: Symmetry breaking.} $\Phi_1$ compares local to global diversity of reachable futures:
\[
\Phi_1(x_i) = H(p_{\text{global}}) - H(p_S)
\]
where $p_{\text{global}} = \sum_i \pi_i T(x_i, \cdot)$ is the global reachable distribution and $p_S = \sum_{j \in S(x_i)} \pi_j T(x_j, \cdot)$ is the local neighborhood distribution ($S$ = $k$-NN in coupling space, $k = 30$).

\subsection*{Locking surface identification}

Locking surfaces were defined as regions of state space where reachable futures have collapsed sufficiently that perturbations fail to reopen alternatives. A cell was classified as ``locked'' if it satisfied three criteria:
\begin{enumerate}
    \item $\Phi_3 < $ 25th percentile (low reachability)
    \item $\Phi_2 > 0$ (stable, not chaotic)
    \item High $\Phi_3$ gradient (steep transition, defined as top 25th percentile of $|\nabla\Phi_3|$ where gradient magnitude was approximated as mean $\Phi_3$ difference to $k$-NN)
\end{enumerate}

\subsection*{Neural network compression of transition operators}

To enable efficient computation of multi-horizon quantities and prospective inference, we trained neural networks to compress the high-dimensional transition operators.

\paragraph{Architecture.} A shared encoder mapped 50-dimensional PCA embeddings to 64-dimensional representations:
\[
\text{PCA}_{50} \rightarrow \text{Linear}_{256} \rightarrow \text{GELU} \rightarrow \text{LayerNorm} \rightarrow \text{Linear}_{256} \rightarrow \text{GELU} \rightarrow \text{LayerNorm} \rightarrow \text{Linear}_{64}
\]
using GELU activations \citep{hendrycks2016} and layer normalization \citep{ba2016}. Learnable time embeddings (one per unique timepoint) were added to cell embeddings before L2 normalization. A linear lock prediction head was attached to source embeddings.

\paragraph{Loss function.} The model was trained with a composite loss:
\[
\mathcal{L} = \mathcal{L}_{\text{NCE}} + 0.25 \cdot \mathcal{L}_{\Phi_3} + 0.25 \cdot \mathcal{L}_{\text{lock}}
\]

\textit{Multi-positive InfoNCE} \citep{oord2018}: For each source cell, we sampled $P = 4$ positive targets weighted by $T$ and $N = 2048$ negatives (50\% hard negatives from $k$-NN, 50\% uniform random):
\[
\mathcal{L}_{\text{NCE}} = -\sum_i \sum_{j \in \text{pos}(i)} T_{ij} \cdot \log \frac{\exp(e_i^\top f_j / \tau)}{\sum_k \exp(e_i^\top f_k / \tau)}
\]
where $\tau = 0.1$ is the temperature and $e_i$, $f_j$ are L2-normalized embeddings.

\textit{$\Phi_3$ rank loss:} To preserve entropy structure, we computed predicted entropy from TopM softmax distributions and matched percentile ranks to true $\Phi_3$ using MSE loss.

\textit{Lock classification:} Binary cross-entropy loss on lock labels from Phase 1.

\paragraph{TopM compression.} The TopM parameter (number of top-ranked targets retained per source) was computed per timepoint pair using a compressibility-based formula:
\[
\text{TopM}_k = \left\lceil \frac{\text{median nnz per row of } T_k}{\tau_{\text{compress}}} \right\rceil
\]
where $\tau_{\text{compress}} = 0.9$ ensures 90\% mass capture. This yielded pair-specific TopM values rather than a global constant.

\paragraph{Training.} Models were trained for 100 epochs using AdamW optimizer \citep{loshchilov2019} with learning rate $10^{-3}$, weight decay $10^{-4}$, batch size 256, and gradient clipping at 1.0. Data was split 80\%/10\%/10\% for train/val/test with deterministic seeding.

\paragraph{Compressed operator construction.} After training, compressed operators $\hat{T}$ were built by:
\begin{enumerate}
    \item Encoding all source cells with source time embedding
    \item Encoding all target cells with target time embedding
    \item Computing pairwise similarities $s_{ij} = e_i^\top f_j / \tau$
    \item For each source, retaining TopM highest similarities
    \item Applying softmax within TopM to obtain row-stochastic probabilities
\end{enumerate}
Results were stored as sparse CSR matrices.

\subsection*{Commitment score definition}

The commitment score $C$ was defined as a multi-horizon measure of future uncertainty collapse:

\paragraph{Multi-horizon entropy.} For each cell at timepoint pair $k$, we computed entropy at horizons $h \in \{1, 2, 3\}$ using composed transition operators:
\[
H^{(h)} = \text{entropy of } (T_k \cdot T_{k+1} \cdots T_{k+h-1}) \text{ row}
\]

\paragraph{Percentile ranks.} Entropies were converted to within-pair percentile ranks $R^{(h)} \in [0, 1]$.

\paragraph{Uncertainty score.} A weighted average combined horizons:
\[
U = \frac{\sum_h w_h R^{(h)}}{\sum_h w_h}, \quad w_h = 1/h
\]
weighting nearer horizons more heavily.

\paragraph{Commitment score.} Finally:
\[
C = 1 - U
\]
such that $C \approx 0$ indicates high competence (many futures) and $C \approx 1$ indicates high commitment (few futures).

\subsection*{Perturbation protocol}

To assess controllability, we applied in silico perturbations to cell states and measured changes in predicted commitment.

\paragraph{Perturbation families.} Three perturbation types were tested:
\begin{enumerate}
    \item \textit{Local random:} Uniform random direction in PCA space
    \item \textit{Directional downC:} Gradient descent direction $-\nabla_x C$ (toward lower commitment)
    \item \textit{Directional upC:} Gradient ascent direction $+\nabla_x C$ (toward higher commitment)
\end{enumerate}

\paragraph{Step size.} Perturbation magnitude was set to $\varepsilon_k = $ median 10-NN distance within the timepoint, ensuring biologically plausible displacements.

\paragraph{Matched nulls.} For each directional perturbation, a matched null was constructed by projecting onto the subspace orthogonal to $\nabla C$, preserving displacement magnitude while removing directionality.

\paragraph{Decile stratification.} Cells were binned by baseline commitment into 10 deciles. Within each decile, we computed:
\[
D = \Delta C_{\text{perturb}} - \Delta C_{\text{null}}
\]
the difference between perturbation effect and matched null effect.

\paragraph{Statistical testing.} Significance was assessed via one-sided Wilcoxon signed-rank tests on paired $(D_{\text{cell}}, 0)$ values within each decile. 95\% confidence intervals on median $D$ were computed via 2,000 bootstrap resamples.

\subsection*{Quantification and Statistical Analysis}

All statistical conclusions were based on effect sizes with empirically-derived null distributions, avoiding arbitrary thresholds. No biological conclusion depends on a fixed cutoff; instead, each test compares observed statistics to matched null distributions.

\paragraph{Permutation tests for correlation.} Spearman correlations (developmental monotonicity, entropy drop rate, $\Phi_3$ concordance) were assessed against permutation null distributions. For each test, the second variable was randomly permuted 10,000 times while preserving the first variable, generating a null distribution of $\rho$ values under the hypothesis of no association. Two-tailed $p$-values were computed as the fraction of permuted $|\rho|$ values exceeding the observed $|\rho|$. This non-parametric approach makes no distributional assumptions and is robust to ties and non-linearity.

\paragraph{Bootstrap confidence intervals.} Confidence intervals for AUROC and perturbation effect sizes were computed via non-parametric bootstrap resampling \citep{efron1979}. For each statistic, 2,000 bootstrap samples were drawn with replacement from the original data, the statistic was recomputed for each resample, and the 2.5th and 97.5th percentiles of the bootstrap distribution defined the 95\% CI. For AUROC, stratified resampling preserved class balance in each bootstrap sample.

\paragraph{AUROC null distribution.} To assess whether lock discrimination exceeded chance, we constructed a shuffled-label null: class labels were randomly permuted 2,000 times, AUROC was computed for each permutation, and $p$-values were computed as the fraction of null AUROCs exceeding the observed. This tests whether discrimination reflects genuine signal rather than class imbalance artifacts.

\paragraph{Perturbation statistical testing.} Perturbation effects were assessed using a paired design: for each cell, directional perturbations were compared to matched null perturbations of identical magnitude but orthogonalized direction. The test statistic $D = \Delta C_{\text{directional}} - \Delta C_{\text{null}}$ was computed for each cell, yielding paired differences that isolate the effect of perturbation direction from magnitude. Significance was assessed via one-sided Wilcoxon signed-rank tests \citep{wilcoxon1945} (testing $D < 0$ for reopening perturbations), which is appropriate for paired non-parametric data with potentially non-normal distributions. Bootstrap 95\% CIs on median $D$ (2,000 resamples) quantified effect size uncertainty.

\paragraph{Matched null perturbation construction.} For each directional perturbation along gradient $\mathbf{g}$, matched nulls were constructed by: (1) sampling a random direction $\mathbf{v} \sim \mathcal{N}(0, I)$; (2) orthogonalizing to the gradient: $\mathbf{v}_\perp = \mathbf{v} - (\mathbf{v}^\top \mathbf{g} / \|\mathbf{g}\|^2)\mathbf{g}$; (3) scaling to identical magnitude: $\mathbf{v}_\perp \leftarrow \varepsilon_k \mathbf{v}_\perp / \|\mathbf{v}_\perp\|$. Five matched nulls were generated per directional perturbation to reduce variance.

\paragraph{Multiple comparisons.} The analysis tested four pre-specified hypotheses per species (monotonicity, entropy drop, lock consistency, $\Phi_3$ concordance) plus stratified perturbation analysis. Given this limited hypothesis set and the use of effect sizes with CIs rather than binary significance, no multiple comparison correction was applied. All correlation tests were two-tailed; perturbation tests were one-sided (directional hypothesis).

\paragraph{Computational implementation.} All analyses were implemented in Python. Key dependencies included Scanpy \citep{wolf2018} for single-cell preprocessing, POT \citep{flamary2021} for optimal transport, PyTorch \citep{paszke2019} for neural network training, and scikit-learn \citep{pedregosa2011} for statistical utilities. Numerical computations used NumPy \citep{harris2020} and SciPy \citep{virtanen2020}. GPU acceleration via CuPy \citep{okuta2017} was used for cost matrix computation, $k$-NN search, and neural network training.

%%%%%%%%%%%%%%%%%%%%%%%%%%%%%%%%%%%%%%%%%%%%%%%%%%%%%%%%%%%%%%%%%%%%%%%%%%%%%%%
% Data and Code Availability
%%%%%%%%%%%%%%%%%%%%%%%%%%%%%%%%%%%%%%%%%%%%%%%%%%%%%%%%%%%%%%%%%%%%%%%%%%%%%%%

\section*{Data and Code Availability}

All code for this study is publicly available at \url{https://github.com/prateek-vadde/differentiation-commitment}. The repository includes complete implementations for transition inference, commitment score computation, perturbation analysis, and figure generation.

Single-cell RNA sequencing data were obtained from published sources: mouse gastrulation data from \citet{pijuansala2019} and zebrafish embryogenesis data from \citet{wagner2018}.

%%%%%%%%%%%%%%%%%%%%%%%%%%%%%%%%%%%%%%%%%%%%%%%%%%%%%%%%%%%%%%%%%%%%%%%%%%%%%%%
% References
%%%%%%%%%%%%%%%%%%%%%%%%%%%%%%%%%%%%%%%%%%%%%%%%%%%%%%%%%%%%%%%%%%%%%%%%%%%%%%%

\bibliographystyle{plainnat}
\bibliography{references}

\end{document}
